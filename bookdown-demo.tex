% Options for packages loaded elsewhere
\PassOptionsToPackage{unicode}{hyperref}
\PassOptionsToPackage{hyphens}{url}
%
\documentclass[
]{book}
\usepackage{lmodern}
\usepackage{amssymb,amsmath}
\usepackage{ifxetex,ifluatex}
\ifnum 0\ifxetex 1\fi\ifluatex 1\fi=0 % if pdftex
  \usepackage[T1]{fontenc}
  \usepackage[utf8]{inputenc}
  \usepackage{textcomp} % provide euro and other symbols
\else % if luatex or xetex
  \usepackage{unicode-math}
  \defaultfontfeatures{Scale=MatchLowercase}
  \defaultfontfeatures[\rmfamily]{Ligatures=TeX,Scale=1}
\fi
% Use upquote if available, for straight quotes in verbatim environments
\IfFileExists{upquote.sty}{\usepackage{upquote}}{}
\IfFileExists{microtype.sty}{% use microtype if available
  \usepackage[]{microtype}
  \UseMicrotypeSet[protrusion]{basicmath} % disable protrusion for tt fonts
}{}
\makeatletter
\@ifundefined{KOMAClassName}{% if non-KOMA class
  \IfFileExists{parskip.sty}{%
    \usepackage{parskip}
  }{% else
    \setlength{\parindent}{0pt}
    \setlength{\parskip}{6pt plus 2pt minus 1pt}}
}{% if KOMA class
  \KOMAoptions{parskip=half}}
\makeatother
\usepackage{xcolor}
\IfFileExists{xurl.sty}{\usepackage{xurl}}{} % add URL line breaks if available
\IfFileExists{bookmark.sty}{\usepackage{bookmark}}{\usepackage{hyperref}}
\hypersetup{
  pdftitle={Sozialwissenschaftliche Datenanalyse mit R},
  pdfauthor={Kenneth Horvath \& Guy Schwegler},
  hidelinks,
  pdfcreator={LaTeX via pandoc}}
\urlstyle{same} % disable monospaced font for URLs
\usepackage{color}
\usepackage{fancyvrb}
\newcommand{\VerbBar}{|}
\newcommand{\VERB}{\Verb[commandchars=\\\{\}]}
\DefineVerbatimEnvironment{Highlighting}{Verbatim}{commandchars=\\\{\}}
% Add ',fontsize=\small' for more characters per line
\usepackage{framed}
\definecolor{shadecolor}{RGB}{248,248,248}
\newenvironment{Shaded}{\begin{snugshade}}{\end{snugshade}}
\newcommand{\AlertTok}[1]{\textcolor[rgb]{0.94,0.16,0.16}{#1}}
\newcommand{\AnnotationTok}[1]{\textcolor[rgb]{0.56,0.35,0.01}{\textbf{\textit{#1}}}}
\newcommand{\AttributeTok}[1]{\textcolor[rgb]{0.77,0.63,0.00}{#1}}
\newcommand{\BaseNTok}[1]{\textcolor[rgb]{0.00,0.00,0.81}{#1}}
\newcommand{\BuiltInTok}[1]{#1}
\newcommand{\CharTok}[1]{\textcolor[rgb]{0.31,0.60,0.02}{#1}}
\newcommand{\CommentTok}[1]{\textcolor[rgb]{0.56,0.35,0.01}{\textit{#1}}}
\newcommand{\CommentVarTok}[1]{\textcolor[rgb]{0.56,0.35,0.01}{\textbf{\textit{#1}}}}
\newcommand{\ConstantTok}[1]{\textcolor[rgb]{0.00,0.00,0.00}{#1}}
\newcommand{\ControlFlowTok}[1]{\textcolor[rgb]{0.13,0.29,0.53}{\textbf{#1}}}
\newcommand{\DataTypeTok}[1]{\textcolor[rgb]{0.13,0.29,0.53}{#1}}
\newcommand{\DecValTok}[1]{\textcolor[rgb]{0.00,0.00,0.81}{#1}}
\newcommand{\DocumentationTok}[1]{\textcolor[rgb]{0.56,0.35,0.01}{\textbf{\textit{#1}}}}
\newcommand{\ErrorTok}[1]{\textcolor[rgb]{0.64,0.00,0.00}{\textbf{#1}}}
\newcommand{\ExtensionTok}[1]{#1}
\newcommand{\FloatTok}[1]{\textcolor[rgb]{0.00,0.00,0.81}{#1}}
\newcommand{\FunctionTok}[1]{\textcolor[rgb]{0.00,0.00,0.00}{#1}}
\newcommand{\ImportTok}[1]{#1}
\newcommand{\InformationTok}[1]{\textcolor[rgb]{0.56,0.35,0.01}{\textbf{\textit{#1}}}}
\newcommand{\KeywordTok}[1]{\textcolor[rgb]{0.13,0.29,0.53}{\textbf{#1}}}
\newcommand{\NormalTok}[1]{#1}
\newcommand{\OperatorTok}[1]{\textcolor[rgb]{0.81,0.36,0.00}{\textbf{#1}}}
\newcommand{\OtherTok}[1]{\textcolor[rgb]{0.56,0.35,0.01}{#1}}
\newcommand{\PreprocessorTok}[1]{\textcolor[rgb]{0.56,0.35,0.01}{\textit{#1}}}
\newcommand{\RegionMarkerTok}[1]{#1}
\newcommand{\SpecialCharTok}[1]{\textcolor[rgb]{0.00,0.00,0.00}{#1}}
\newcommand{\SpecialStringTok}[1]{\textcolor[rgb]{0.31,0.60,0.02}{#1}}
\newcommand{\StringTok}[1]{\textcolor[rgb]{0.31,0.60,0.02}{#1}}
\newcommand{\VariableTok}[1]{\textcolor[rgb]{0.00,0.00,0.00}{#1}}
\newcommand{\VerbatimStringTok}[1]{\textcolor[rgb]{0.31,0.60,0.02}{#1}}
\newcommand{\WarningTok}[1]{\textcolor[rgb]{0.56,0.35,0.01}{\textbf{\textit{#1}}}}
\usepackage{longtable,booktabs}
% Correct order of tables after \paragraph or \subparagraph
\usepackage{etoolbox}
\makeatletter
\patchcmd\longtable{\par}{\if@noskipsec\mbox{}\fi\par}{}{}
\makeatother
% Allow footnotes in longtable head/foot
\IfFileExists{footnotehyper.sty}{\usepackage{footnotehyper}}{\usepackage{footnote}}
\makesavenoteenv{longtable}
\usepackage{graphicx,grffile}
\makeatletter
\def\maxwidth{\ifdim\Gin@nat@width>\linewidth\linewidth\else\Gin@nat@width\fi}
\def\maxheight{\ifdim\Gin@nat@height>\textheight\textheight\else\Gin@nat@height\fi}
\makeatother
% Scale images if necessary, so that they will not overflow the page
% margins by default, and it is still possible to overwrite the defaults
% using explicit options in \includegraphics[width, height, ...]{}
\setkeys{Gin}{width=\maxwidth,height=\maxheight,keepaspectratio}
% Set default figure placement to htbp
\makeatletter
\def\fps@figure{htbp}
\makeatother
\setlength{\emergencystretch}{3em} % prevent overfull lines
\providecommand{\tightlist}{%
  \setlength{\itemsep}{0pt}\setlength{\parskip}{0pt}}
\setcounter{secnumdepth}{5}
\usepackage{booktabs}
\usepackage{amsthm}
\makeatletter
\def\thm@space@setup{%
  \thm@preskip=8pt plus 2pt minus 4pt
  \thm@postskip=\thm@preskip
}
\makeatother
\usepackage[]{natbib}
\bibliographystyle{apalike}

\title{Sozialwissenschaftliche Datenanalyse mit R}
\author{Kenneth Horvath \& Guy Schwegler}
\date{2020-09-07}

\begin{document}
\maketitle

{
\setcounter{tocdepth}{1}
\tableofcontents
}
\hypertarget{einfuxfchrung}{%
\chapter*{Einführung}\label{einfuxfchrung}}
\addcontentsline{toc}{chapter}{Einführung}

Das Seminar ``Sozialwissenschaftliche Datenanalyse mit R'' bietet eine systematische Einführung in das Statistikpaket R. R ist eine Open Source Software, die sich unter anderem durch Flexibilität und vielfältige Möglichkeiten der grafischen und numerischen Datenanalyse auszeichnet. Das Seminar führt in inhaltlicher Abstimmung mit der Vorlesung „Grundlagen der multivariaten Statistik`` in Aufbau und Funktionsweise des Programms sowie in die Umsetzung wichtiger statistischer Verfahren (etwa lineare Regression und logistische Regression) ein. Anhand dieser Verfahren werden unter anderem Techniken des effizienten Datenmanagements, Möglichkeiten, eigenständig kleine Funktionen zu programmieren, sowie Formen der grafischen Datenanalyse und Ergebnisdarstellung besprochen.

Das vorliegende Dokument ist ein sogennantes Bookdown (\citet{R-bookdown}, siehe auch \href{https://bookdown.org/}{hier}) und dient der Ergebnissicherung im Seminarverlauf. Das heisst dass die im Seminar besprochene Themen hiernochmals diskutiert, Lösungen präsentiert und mit Literatur ergäntz werden (siehe für allgemeine Literatur etwa \citet{DiazBone2019}, \citet{Kabacoff2015} oder \citet{Manderscheid2017}). Das Bookdown wird laufend aktualisiert.

\hypertarget{einheit-01-17.09.}{%
\chapter{Einheit 01 (17.09.)}\label{einheit-01-17.09.}}

\hypertarget{sozialwissenschaftliche-datenanalyse}{%
\section{Sozialwissenschaftliche Datenanalyse}\label{sozialwissenschaftliche-datenanalyse}}

Realität des statistischen Arbeitens (Seminar im Gegensatz zur Vorlesung)
Modell
Import
Tidy und Transform (mit Fragebogen und Datensatz umgehen/nützlich machen)
Techniken für einen Überblick über die Daten (gerade speziell bei R)
Model (Techniken zur Analyse)
Visualise
Communicate (Techniken zum Arbeiten/Darstellen/Exportieren (R Markdown)

\hypertarget{r-als-programmprogrammiersprache}{%
\section{R als Programm/Programmiersprache}\label{r-als-programmprogrammiersprache}}

\begin{enumerate}
\def\labelenumi{\arabic{enumi}.}
\tightlist
\item
  State of The Art
  Anfang der 90er Jahre entwickelt an der Universität Auckland in Neuseeland (Gegensatz SPSS, SPSS als Relikt), immer das aktuellste, mehr Möglichkeiten, insbesondere grafische Analysemethoden, aber auch QDA ist möglich, tolle Grafiken/Tabellen, (\ldots)
  Entwicklung im Austausch
  Schnell (Big Data), Flexibel (immer nur das gebraucht, was gebraucht werden muss, verschiedene GUI kostenlos
  negativ Punkt: keine `fertiges' Programm
\item
  Open Sourse, riesige Anzahl Package (fast 12'000)
\item
  GUIs
  GUIs zeigen -\textgreater{} R Konsole, R Commander, R Studio
\end{enumerate}

\hypertarget{ziel-des-kurses}{%
\section{Ziel des Kurses}\label{ziel-des-kurses}}

Arbeiten mit R (Einführung in die Struktur und Funktionsweise), Datenanalyse (Anwendung von bestimmten Methoden  grafische Verfahren) und konkrete Analyse (Umgang mit Datensätzen, Ergebnisse produzieren)
damit eigene Forschungsprojekte durchgeführt werden können
Anforderungen
aktives mitmachen, Falllösungen
und Ablauf Seminar
Syllabus
`sehr sportlich', wird wohl nicht ganz so durchgeführt werden können
Vorlesung
Seminar läuft mehr oder weniger parallel zur Vorlesung, am Anfang Inferenzstatistik, dann Einführung in R und `einfachere' Arbeitsschritte, dann die Methoden der Vorlesung
Seminar ist keine Übung zur Vorlesung, sondern Anwendung und Veranschaulichung
Daten
Eigen(s) generierte Daten und ESS

\hypertarget{installation-r-ausblick}{%
\section{Installation R, Ausblick}\label{installation-r-ausblick}}

Wer hat alles schon installierte?
Installieren, andere wieder mal reinkommen, auch Kapitel 1+2+3 aus Manderscheids Buch
Ziel für nächstes Mal: Installiert und auch mit dem ESS vertraut gemacht

\hypertarget{einheit-01-24.09.}{%
\chapter{Einheit 01 (24.09.)}\label{einheit-01-24.09.}}

\hypertarget{einfuxfchrung}{%
\section{Einführung}\label{einfuxfchrung}}

Starten Sie R Studio (als erster Arbeitsauftrag). Falls Sie Probleme bei der Installation und ähnlichem hatten sollten Sie das weiterhin melden.

Insbesondere in der heutigen Einheit, aber auch anschliessend im weiteren Seminarverlauf, geht es um die Vermittlung zwischen \textbf{Trivialität} und \textbf{Komplexität}: Das heisst es gilt die kleinen Schritte ernstzunehmen, sonst werden die grossen Schritte sehr schnell mühsam.R als Software und als Programmiersprache hat eine steile Lernkurve und viele Probleme werden zu Beginn auftauchen. Mit und durch dise Problemen soll allerdings auch eine eigene Arbeitsweise mit dem Programm gelernt werden.

Weiter gilt es im Lernprozess \textbf{R als Sprache} aufzufassen: Am Anfangs wird man vieles nicht verstehen, später aber dann immer mehr und mehr, man wird sicher. Aber das wichtigste dabei ist, dass man eine Sprache erst dadurch lernt, wenn man sie immer wieder anwendet.

\begin{center}\rule{0.5\linewidth}{0.5pt}\end{center}

\hypertarget{repetition}{%
\section{Repetition}\label{repetition}}

Vier Fenster/Bereiche in R Studio:

\begin{itemize}
\item
  \begin{enumerate}
  \def\labelenumi{\arabic{enumi}.}
  \tightlist
  \item
    Console: Konsole=R, ohne grafischen Filter. Befehle werden über die Konsole gespielt, in der Konsole läuft R.
  \end{enumerate}
\item
  \begin{enumerate}
  \def\labelenumi{\arabic{enumi}.}
  \setcounter{enumi}{1}
  \tightlist
  \item
    R-Skript: Befehle in einem Text schreiben. Das Skript Fenster ist einfach eine Text Datei. Skript als ``Arbeitsgedächtnis'' -- alle Befehle können wieder aufgerufen werden. Und man kann
    kommentieren mittels eines \textbf{\#}.
  \end{enumerate}
\end{itemize}

--\textgreater{} Taschenrechner in Konsole und Taschenrechner in Skript
Hinweis auf das Prompt:
Das \textbf{\textgreater{}} heisst, ich bin ready, frag mich was!

Im Skript oder in der Konsole:

\begin{Shaded}
\begin{Highlighting}[]
\DecValTok{1}\OperatorTok{+}\DecValTok{1}
\DecValTok{5} \OperatorTok{+}\StringTok{ }\DecValTok{3} \OperatorTok{-}\StringTok{ }\DecValTok{4}
\NormalTok{(}\DecValTok{5} \OperatorTok{+}\StringTok{ }\DecValTok{3} \OperatorTok{-}\StringTok{ }\DecValTok{4}\NormalTok{) }\OperatorTok{/}\StringTok{ }\DecValTok{5}
\NormalTok{x <-}\StringTok{ }\NormalTok{(}\DecValTok{5} \OperatorTok{+}\StringTok{ }\DecValTok{3} \OperatorTok{-}\StringTok{ }\DecValTok{4}\NormalTok{) }\OperatorTok{/}\StringTok{ }\DecValTok{5} \CommentTok{#hier der Hinweis auf die Pfeiltaste}
\end{Highlighting}
\end{Shaded}

Mit diesem Zusatz « \textless- » wird x als Objekt definiert (siehe unten), in der Environment abgespeichert und
kann verwendet werden.

\begin{itemize}
\item
  \begin{enumerate}
  \def\labelenumi{\arabic{enumi}.}
  \setcounter{enumi}{2}
  \tightlist
  \item
    Environment: Da werden alle Formen von Daten abgespeichert, die Grundlagen für die Befehle sind. Ebenfalls findet man dort eine History seiner Befehle.
  \end{enumerate}
\end{itemize}

\begin{Shaded}
\begin{Highlighting}[]
\NormalTok{x}
\end{Highlighting}
\end{Shaded}

\begin{verbatim}
## [1] 0.8
\end{verbatim}

\begin{itemize}
\item
  \begin{enumerate}
  \def\labelenumi{\arabic{enumi}.}
  \setcounter{enumi}{3}
  \tightlist
  \item
    Was fehlt jetzt noch? Ergebnis! Grafiken, Hilfe, etc. I Fenster unten rechts teilt uns R-Studio etwas mit.
  \end{enumerate}
\end{itemize}

\begin{center}\rule{0.5\linewidth}{0.5pt}\end{center}

\hypertarget{rmarkdown}{%
\section{RMarkdown}\label{rmarkdown}}

Man kann entweder mit der Konsole, mit dem Skript oder als dritte Möglichkeit auch mit dem \textbf{Markdown} arbeiten. Eine neue Markdown Datei kann via «File\textgreater New File\textgreater R Markdown» oder dem Button direkt unter dem «File» (oder «Datei») Button geöffnet werden.

Markdown ist unterteil in einfachen Fliesstext (weisser Hintergrund) oder R Chunk, d.h. heisst Code Stücke (Befehle) für R. Ein R Chunk kann entweder über den «Insert»-Button im Markdown-Fenster eingefügt werden, oder aber über die Tastenkombination «Ctrl + Alt + i» (bzw. «Cmd + Option + I» bei Mac).

Vorteil am Arbeiten mit Markdown ist
* 1 eine bessere Übersicht/Struktur und Eingabe,
* 2. dass via dem «Knit»-Befehle ein Dokument (wie dieses hier) erstellt werden kann und
* 3. Ergebnisse bei jeden `knitten' neu geladen werden (so sind die ausgegebenen Dokument immer aktuell).

\begin{quote}
Für die verschiedenen \emph{Formatierungsmöglichkeiten} siehe R-Markdown References im OLAT-Ordner `Literatur/R-Cheatsheets'.
\end{quote}

\begin{center}\rule{0.5\linewidth}{0.5pt}\end{center}

\hypertarget{funktionen-und-objekte}{%
\section{Funktionen und Objekte}\label{funktionen-und-objekte}}

R als Sprache besteht aus zwei Elementen
* 1. Funktionen
* 2. Objekte

\hypertarget{funktionen}{%
\subsection{Funktionen}\label{funktionen}}

Funktionen sind etwas, das R macht; wir sagen R, es soll etwas machen. Sie legen fest, was passieren soll und sind immer über die beiden Klammern definiert.

\begin{Shaded}
\begin{Highlighting}[]
\KeywordTok{sqrt}\NormalTok{(x)}
\end{Highlighting}
\end{Shaded}

\begin{verbatim}
## [1] 0.8944272
\end{verbatim}

Funktionen führen etwas aus mit einem \emph{Objekt}, und spezifizieren wie etwas gemacht wird über \emph{Argumente}.

Auch wenn wir uns Hilfe holen möchten, können wir das über eine Hilfe Funktion tun:

\begin{Shaded}
\begin{Highlighting}[]
\KeywordTok{help}\NormalTok{(}\StringTok{"sqrt"}\NormalTok{) }\CommentTok{#hier mit Anführungs- und Schlusszeichen}
\CommentTok{#oder eine andere Variante:}
\NormalTok{?sqrt }\CommentTok{#ohne Anführungs- und Schlusszeichen}
\NormalTok{?mean}

\KeywordTok{help}\NormalTok{(}\StringTok{"help"}\NormalTok{) }
\NormalTok{?help}
\end{Highlighting}
\end{Shaded}

Diese Funktionen sind in den verschiedensten Paketen organisiert. Auch RStudio und RMarkdown sind eigentlich nichts anderes als Pakate. Pakete werden installiert mit der Funktion install.packages() und dann geladen, wenn nötig, mittels library(). Pakete sind daher wie Bücher, die man sich besorgt hat und die ordentlich in einem Regal sind, bis man sie zum verwenden möchte.

Mit folgender Funktion können wir etwa betrachten, welche Pakate installiert worden sind:

\begin{Shaded}
\begin{Highlighting}[]
\KeywordTok{installed.packages}\NormalTok{()}
\end{Highlighting}
\end{Shaded}

\hypertarget{objekte}{%
\subsection{Objekte}\label{objekte}}

R ist eine Objekt-basierte Programmiersprache. Das heisst wir arbeiten immer mit Objekten, und alles kann eigentlich ein Objekte werden. Wenn wir nach Hilfe suchen mittels:

\begin{Shaded}
\begin{Highlighting}[]
\KeywordTok{help}\NormalTok{(}\StringTok{"help"}\NormalTok{)}
\end{Highlighting}
\end{Shaded}

Dann wird die Funktion ``help'' zum Objekte der help-Funktion.

\ldots oder wir nehmen einfach unser Objekt x von oben und machen nochmals was damit:

\begin{Shaded}
\begin{Highlighting}[]
\NormalTok{x }\OperatorTok{+}\StringTok{ }\DecValTok{3}
\end{Highlighting}
\end{Shaded}

\begin{verbatim}
## [1] 3.8
\end{verbatim}

\begin{Shaded}
\begin{Highlighting}[]
\NormalTok{x }\OperatorTok{*}\StringTok{ }\NormalTok{x}
\end{Highlighting}
\end{Shaded}

\begin{verbatim}
## [1] 0.64
\end{verbatim}

Je nach Zeit: Definition des Arbeitsverzeichnis
\ldots entweder mittels:

\begin{Shaded}
\begin{Highlighting}[]
\CommentTok{#setwd()}
\end{Highlighting}
\end{Shaded}

\ldots das ist aber unglaublich mühsam\ldots{}
\ldots und funktioniert nur beschränkt im Markdown-File selber.
Weiter als Hinweis zu Blackslash und Slash:
Kopiert man den Pfad des Ordners via Windows erscheinen Backslashes ``" anstatt (wie von R verlangt)
Slashes''/". Dies kann manchmal zu Problemem führen.

Darum sollten Sie dies als \textbf{absolute Ausnahme} via der \emph{Menüsteuerung} machen:
\emph{Session\textgreater Set Working Directory\textgreater Choose Directory}
Wählen Sie dann eine Ordner aus\ldots{} und kopieren Sie die Code Zeile aus der Konsole als Hinweis in ihre Markdown Date hinein.

Oder geben Sie den Befehl setwd(``PfadzuihremAArbeitsordner'') in der Konsole ein und kopieren Sie diesen.

\emph{Hier habe ich die Definition des Arbeitsordners als Hinweis drin}

\begin{Shaded}
\begin{Highlighting}[]
\KeywordTok{setwd}\NormalTok{(}\StringTok{"C:/Users/SchweglG/R_Daten/HS19/E2"}\NormalTok{)}
\end{Highlighting}
\end{Shaded}

Und dann als kleiner Versuch kann man sich mal die Liste seiner Pakete als CSV-Datei ausgeben lassen. CSV (\textbf{C}omma \textbf{S}eperated \textbf{V}alues) ist ein sehr kleines und sehr flexibles Datenformat, dass -- wie es der Name bereits sagt -- einfach die einzelne Werte mittels Kommas und Zeilen in Datentabellen unterteilt.

\begin{Shaded}
\begin{Highlighting}[]
\NormalTok{pakete_liste <-}\StringTok{ }\KeywordTok{installed.packages}\NormalTok{()}
\KeywordTok{ls}\NormalTok{()}
\end{Highlighting}
\end{Shaded}

\begin{verbatim}
## [1] "pakete_liste" "x"
\end{verbatim}

\begin{Shaded}
\begin{Highlighting}[]
\KeywordTok{write.table}\NormalTok{(pakete_liste, }\DataTypeTok{file=}\StringTok{"Paketliste_neu.csv"}\NormalTok{, }\DataTypeTok{sep=}\StringTok{","}\NormalTok{) }\CommentTok{#CSV Comma-Seperated-Value}
\end{Highlighting}
\end{Shaded}

Was haben wir hier gemacht? Wir machen unsere Lise mit Paketen als ein Objekt abgespeichert (das Objekt `pakete\_liset') und dieses haben wir uns auf CSV-Datei abspeichern lassen. Und da wir keine weiteren Spezifikationen angegeben haben, speichert R die Daten in unser Arbeitsverzeichnis ab.

\hypertarget{arten-von-objekten}{%
\section{Arten von Objekten}\label{arten-von-objekten}}

R kennt nun verschiedenste Arten von Objekten, wir bleiben heute mal bei den Zahlen.
Auch hier gibt es
* einzelne Zahlen
* eine Reihe von Zahlen (Vektoren)
* Matrizen von Zahlen (Matrix)
sowie
* Dataframes («Datensätze», werden wir später noch kennenlernen)
* Listen \& Arrays (lassen wir hier mal weg)

\hypertarget{einzelne-zahlen}{%
\subsection{einzelne Zahlen}\label{einzelne-zahlen}}

\begin{Shaded}
\begin{Highlighting}[]
\NormalTok{x}
\end{Highlighting}
\end{Shaded}

\begin{verbatim}
## [1] 0.8
\end{verbatim}

\begin{Shaded}
\begin{Highlighting}[]
\KeywordTok{rm}\NormalTok{(x)}
\KeywordTok{ls}\NormalTok{()}
\end{Highlighting}
\end{Shaded}

\begin{verbatim}
## [1] "pakete_liste"
\end{verbatim}

Vorschlag: Wir lassen uns immer den gesamten Working-Space -- das heisst all unsere Objekte in der Environment -- automatisch löschen, nachdem wir R beenden. Das hat den Vorteil, das ein Codes gründlich sein muss und nicht etwaige Objekte noch irgendwo ``hängen'' bleiben.
\emph{Tools\textgreater Global Options\textgreater General\textgreater Workspace\textgreater Save Workspace to .Rdate on exit: Never}

Zurück zu unserem einzelnen Zahlen

\begin{Shaded}
\begin{Highlighting}[]
\NormalTok{x <-}\StringTok{ }\DecValTok{1}
\KeywordTok{is.numeric}\NormalTok{(x) }\CommentTok{#Hinweis auf Tabulatortaste}
\end{Highlighting}
\end{Shaded}

\begin{verbatim}
## [1] TRUE
\end{verbatim}

\begin{Shaded}
\begin{Highlighting}[]
\NormalTok{v <-}\StringTok{ }\DecValTok{2}
\NormalTok{w <-}\StringTok{ }\DecValTok{3}
\NormalTok{y <-}\StringTok{ }\DecValTok{2}
\NormalTok{z <-}\StringTok{ }\DecValTok{1}
\NormalTok{v }\OperatorTok{+}\StringTok{ }\NormalTok{w }\OperatorTok{+}\StringTok{ }\NormalTok{x }\OperatorTok{+}\StringTok{ }\NormalTok{y }\OperatorTok{+}\StringTok{ }\NormalTok{z}
\end{Highlighting}
\end{Shaded}

\begin{verbatim}
## [1] 9
\end{verbatim}

x, y, z sind jetzt alles einzelne Zahlen. Das ist natürlich nicht sonderlich interessant, dann wir operieren ja meisten mittels mehreren Zahlen\ldots{}

\hypertarget{vektoren}{%
\subsection{Vektoren}\label{vektoren}}

Ein Vektor in R bezeichnet eine Menge von Elementen und entspricht nicht dem mathematischen
Konzept eines Vektors. Vektoren können sowohl eine Menge an Zahlen als auch an Buchstaben bzw. Zeichen enthalten (wir bleiben vorerst bei Zahlen).

\begin{Shaded}
\begin{Highlighting}[]
\NormalTok{v1 <-}\StringTok{ }\KeywordTok{c}\NormalTok{(}\DecValTok{176}\NormalTok{,}\DecValTok{180}\NormalTok{,}\DecValTok{192}\NormalTok{,}\DecValTok{156}\NormalTok{,}\DecValTok{168}\NormalTok{)}
\KeywordTok{is.vector}\NormalTok{(v1)}
\end{Highlighting}
\end{Shaded}

\begin{verbatim}
## [1] TRUE
\end{verbatim}

Das ähnelt nun schon fast einer Variable - z.Bsp. Körpergrösse mit 7 Fällen. Bennen wir doch daher unseren neuen vektor, der eine Variable sein könnte, auch dementsprechend:

\begin{Shaded}
\begin{Highlighting}[]
\NormalTok{koerp_g <-}\StringTok{ }\NormalTok{v1}
\NormalTok{koerp_g}
\end{Highlighting}
\end{Shaded}

\begin{verbatim}
## [1] 176 180 192 156 168
\end{verbatim}

Wir können natürlich auch einzeln Objekte zu einem neuen Vektorhinzufügen:

\begin{Shaded}
\begin{Highlighting}[]
\NormalTok{gndr <-}\StringTok{ }\KeywordTok{c}\NormalTok{(v,w,x,y,z)}
\NormalTok{gndr}
\end{Highlighting}
\end{Shaded}

\begin{verbatim}
## [1] 2 3 1 2 1
\end{verbatim}

\begin{Shaded}
\begin{Highlighting}[]
\KeywordTok{is.vector}\NormalTok{(gndr)}
\end{Highlighting}
\end{Shaded}

\begin{verbatim}
## [1] TRUE
\end{verbatim}

\hypertarget{matrizen}{%
\subsection{Matrizen}\label{matrizen}}

\begin{Shaded}
\begin{Highlighting}[]
\KeywordTok{rbind}\NormalTok{(koerp_g,gndr)}
\end{Highlighting}
\end{Shaded}

\begin{verbatim}
##         [,1] [,2] [,3] [,4] [,5]
## koerp_g  176  180  192  156  168
## gndr       2    3    1    2    1
\end{verbatim}

\begin{Shaded}
\begin{Highlighting}[]
\NormalTok{data <-}\StringTok{ }\KeywordTok{rbind}\NormalTok{(koerp_g,gndr) }\CommentTok{#rbind ist eine Funktion, wo wir die Reihen (bzw. Zeilen) aneinanderfügen}
\KeywordTok{is.matrix}\NormalTok{(data)}
\end{Highlighting}
\end{Shaded}

\begin{verbatim}
## [1] TRUE
\end{verbatim}

Eine Matrix bzw. Matrize enthalten Zeilen und Spalten. Dabei können sie nur Daten eines Typs, also entweder numerische oder Textdaten, kombinieren\ldots. Trotzdem haben wir so uns einen ersten kleinen Datensatz erstellt, indem (z.Bsp.) Körpergrösse und Geschlecht enthalten ist.

\hypertarget{arbeitsauftrag}{%
\section{Arbeitsauftrag}\label{arbeitsauftrag}}

Berechnen Sie für den folgenden Vektor in R Mittelwert und Standardabweichung, ohne auf die in R eingebauten Funktionen mean() {[}berechnet das arithmetische Mittel eines Vektors{]} und sd() {[}berechnet die Standardabweichung eines Vektors{]} zurückzugreifen -- also quasi `von Hand'.

Vektor: (1, 3, 7, 9, 22, 2, 8, 14, 20, 3, 7, 9, 11, 13)

Vergleichen Sie Ihr Ergebnis mit jenem der in R eingebauten Funktionen mean() und sd().

Falls es Abweichungen zwischen Ihrer Lösung und derjenigen von R gibt, woran könnte das liegen?

Halten Sie Ihren R-Code und Ihre Erläuterungen, Fragen, Probleme etc. in einem R-Markdown-Dokument fest. `Knitten' Sie dieses zu einem PDF und laden Sie die Datei via OLAT hoch. Sollten Sie Probleme mit dem Markdown haben können Sie im R-Skript arbeiten, ihre geschriebenen Zeilen rauskopieren und ggfs. in Word (oder ähnlichem) ergänzen.

\hypertarget{methods}{%
\chapter{Methods}\label{methods}}

We describe our methods in this chapter.

\hypertarget{applications}{%
\chapter{Applications}\label{applications}}

Some \emph{significant} applications are demonstrated in this chapter.

\hypertarget{example-one}{%
\section{Example one}\label{example-one}}

\hypertarget{example-two}{%
\section{Example two}\label{example-two}}

\hypertarget{final-words}{%
\chapter{Final Words}\label{final-words}}

We have finished a nice book.

  \bibliography{book.bib,packages.bib}

\end{document}
